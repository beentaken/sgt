\hsize=5in \hoffset=0.5in
\vsize=100in
\font\rm=cmr20
\nopagenumbers

\font\ms=cmsy20
\chardef\zeroeff="0F
\def\tbullet{{\ms\zeroeff}}
\font\mi=cmmi20
\chardef\threeeff="3F
\def\tstar{{\mi\threeeff}}

% We expect a number on standard input giving the mode in which this
% file is to be rendered:
%
%  - Mode 0 gives the base image for the DVD menu.
%  - Mode 1 draws rule boxes around all the menu items, so that dvitype
%    can be used to extract the rule positions and determine the menu
%    items' bounding rectangles.
%  - Mode 2 draws the menu items highlighted.
%  - Mode 3 draws the menu items selected.
\read16 to\type

% In mode 1, \menuitem replaces the item box with a rule.
\ifnum\type=1
\def\menuouter#1{
\setbox1=\hbox{#1}
\vbox{\hrule width\wd1 height\ht1 depth\dp1}
}
\else
\def\menuouter#1{
\setbox1=\hbox{#1}
\vbox{\hbox to\wd1{\copy1}}
}
\fi

\def\menuitem#1{\menuouter{\menuinner{#1}}}

% Actually displaying the menu item is done differently depending on
% mode.
%
% In mode 0 (unhighlighted) we display the actual item text.
% In mode 1 (highlighted) we omit the item text (to avoid overlaying
% it on itself), but instead display a bullet to its left.
% In mode 2 (selected) we replace the bullet with a star.

\ifnum\type=2
\def\menuinner#1{\hbox{\hbox to0em{}\hbox to 1.2em{\hfill\tbullet\hfill}\phantom{#1}\hbox to0em{}}}
\else
\ifnum\type=3
\def\menuinner#1{\hbox{\hbox to0em{}\hbox to 1.2em{\hfill\tstar\hfill}\phantom{#1}\hbox to0em{}}}
\else
\def\menuinner#1{\hbox{\hbox to0em{}\hbox to 1.2em{\hfill}\hbox{#1}\hbox to0em{}}}
\fi
\fi

\def\titleline#1{\line{\hfill#1\hfill}}

\def\title#1{\vbox to0pt{\phantom{x}\vskip1in#1\vss}}

\def\menu#1{\vskip2.5in\line{\hfill \vbox{#1}\hfill}}

\rm
